%% ============================================
%% 115年度教育部教學實踐研究計畫
%% 肆、研究設計與執行規劃
%% ============================================

\subsection{研究設計與執行規劃}

\subsubsection{研究設計}

本研究採用\textbf{前後測對比設計},整合量化與質性資料\cite{creswell2018research},探討AI鷹架漸退教學模式對學生學習成效之影響。

\subsubsection{研究問題}

\begin{enumerate}
\item \textbf{學習成效}:學生的Python程式設計能力是否顯著提升?
\item \textbf{行為轉變}:學生對AI的依賴程度與自我調節學習能力如何轉變?
\item \textbf{歷程分析}:AI鷹架設計如何影響學生的學習策略?
\item \textbf{學習者觀點}:學生對AI輔助學習的態度與反思為何?
\end{enumerate}

\subsubsection{研究架構}

\begin{itemize}[leftmargin=2em]
\item \textbf{自變項}:三階段AI鷹架教學(引導期→轉換期→自主期)
\item \textbf{依變項}:程式設計能力、AI依賴程度、自我調節學習能力
\item \textbf{控制變項}:教學內容、授課教師、上課時數
\end{itemize}

\subsubsection{研究對象}

國立東華大學115學年度第二學期「初級程式設計-Python」通識課程修課學生,預估50--60人。研究對象為非資訊科系學生,無程式設計經驗或僅有基礎經驗。

\subsubsection{研究工具}

本研究使用以下六項工具蒐集資料:

\begin{table}[htbp]
\centering
\footnotesize
\caption{研究工具一覽}
\label{tab:research_tools}
\renewcommand{\arraystretch}{1.4}
\begin{tabular}{|L{2cm}|L{3.8cm}|L{3.5cm}|}
\hline
\textbf{工具} & \textbf{內容說明} & \textbf{施測時間} \\
\hline
Python能力測驗 & 選擇題與程式撰寫題 & 第2週(前測)、第17週(後測) \\
\hline
SRL與AI使用量表 & 短版量表16--20題 & 第6、12、17週 \\
\hline
PBL專題評量 & 程式正確性、問題分析等五向度 & 第12週、第18週 \\
\hline
AI互動日誌 & 問題類型、嘗試步驟、採納理由 & 每週3--5筆 \\
\hline
學習反思報告 & 150--200字,「一錯一改」架構 & 每4週1次 \\
\hline
焦點團體訪談 & 每場6--8人,約60分鐘 & 第12週(2場) \\
\hline
\end{tabular}
\end{table}

\subsubsection{資料蒐集時程}

\begin{table}[htbp]
\centering
\footnotesize
\caption{資料蒸集時程}
\label{tab:data_collection}
\renewcommand{\arraystretch}{1.4}
\begin{tabular}{|c|L{4.5cm}|L{3cm}|}
\hline
\textbf{週次} & \textbf{資料類型} & \textbf{對應研究問題} \\
\hline
第2週 & Python前測 & 學習成效 \\
\hline
第6週 & 第一次量表、反思報告 & 行為轉變 \\
\hline
第12週 & 第二次量表、訪談、期中專題 & 行為、歷程、觀點 \\
\hline
第17週 & Python後測、第三次量表 & 學習成效、行為轉變 \\
\hline
第18週 & 期末專題、反思報告 & 學習成效、觀點 \\
\hline
全學期 & AI互動日誌 & 歷程分析 \\
\hline
\end{tabular}
\end{table}

\subsubsection{資料分析方法}

\textbf{量化分析}:
\begin{itemize}[leftmargin=2em]
\item 前後測比較:比較學期初與學期末成績差異
\item 三時點量表:分析第6、12、17週量表分數變化趨勢
\item AI互動行為:統計學生「先嘗試後求助」的比例變化
\end{itemize}

\textbf{質性分析}:
採用主題分析法\cite{braun2006thematic},聚焦自我調節、AI依賴、學習策略三主題,雙人編碼確保信度。

\subsubsection{研究倫理}

\begin{itemize}[leftmargin=2em]
\item \textbf{知情同意}:開學第一週說明研究目的,學生簽署同意書
\item \textbf{資料保護}:所有資料去識別化,僅供學術研究
\item \textbf{AI使用揭露}:明確告知使用規範與學術誠信要求\cite{becker2023programming}
\end{itemize}
