%% ============================================
%% 115年度教育部教學實踐研究計畫
%% 肆、研究設計與執行規劃
%% ============================================

\section{研究設計與執行規劃}

\subsection{研究設計}

本研究採用\textbf{單組前後測準實驗設計}(one-group pretest-posttest quasi-experimental design),參考研究方法學相關論述\cite{creswell2018research},並以混合方法整合量化與質性資料。

\subsubsection{研究問題}

本研究針對以下四個核心問題進行探討:

\begin{enumerate}
\item \textbf{研究問題一(學習成效)}:在三階段AI鷹架漸退模式下,學生的Python程式設計能力是否顯著提升?
\item \textbf{研究問題二(學習行為轉變)}:學生對AI工具的依賴程度與自我調節學習能力,在三階段教學中如何轉變?
\item \textbf{研究問題三(學習歷程分析)}:不同階段的AI鷹架設計,如何影響學生的學習策略與問題解決歷程?
\item \textbf{研究問題四(學習者觀點)}:學生對AI輔助程式設計學習的態度、使用經驗與學習反思為何?
\end{enumerate}

\subsubsection{研究架構}
\begin{itemize}[leftmargin=2em]
\item \textbf{自變項}:三階段AI鷹架教學介入(引導期→轉換期→自主期)
\item \textbf{依變項}:Python程式設計能力、AI依賴程度、自我調節學習能力、學習行為模式
\item \textbf{控制變項}:教學內容、授課教師、上課時數
\end{itemize}

\subsection{研究對象}

國立東華大學114學年度第二學期「程式設計與應用」通識課程修課學生,預估50-60人。研究對象為非資訊科系學生,來自人文、理工、管理等不同學院,無程式設計經驗或僅有基礎經驗。

\subsection{研究工具}

\subsubsection{工具一:Python程式設計能力測驗}

測驗架構參考CS50P課程設計與內容\cite{cs50p2024},包含選擇題與程式撰寫題,涵蓋理解、應用、分析、創造四個認知層次。僅描述施測方法與題型設計原則,本計畫不進行效度、預試與信度結果報告。

\subsubsection{工具二:AI依賴與自我調節學習量表}

包含AI使用態度分量表(12題)與SRL能力分量表(23題),採用Likert 5點量表。僅描述題構與施測流程,本計畫不報告建構效度與信度數值。

\textbf{施測時間}:第6週(引導期末)、第12週(轉換期末)、第17週(自主期末),共三次施測。

\subsubsection{工具三:PBL專題評量規準}

評量向度:程式碼正確性(30\%)、問題分析深度(25\%)、介面設計(20\%)、創意應用(15\%)、學習反思(10\%)。僅說明評分構面與規準設計,本計畫不報告評分者間信度數值。

\subsubsection{工具四:AI互動紀錄}

學生互動資料之分析與分類,參考既有研究提出的架構\cite{prather2024weird}。

\subsubsection{工具五:學習反思報告}

每4週撰寫1次(共4次),質性分析採用主題分析法\cite{braun2006thematic}。

\subsubsection{工具六:焦點團體訪談}

第12週進行2場訪談,每場6-8人,約60分鐘。

\subsection{資料蒐集時程}

\begin{table}[h]
\centering
\caption{資料蒐集時程與研究問題對應}
\renewcommand{\arraystretch}{1.25}
\begin{tabular}{|l|m{5.5cm}|m{5cm}|}
\hline
\textbf{時間} & \textbf{資料類型} & \textbf{對應研究問題} \\
\hline
第2週 & Python前測 & 研究問題一:學習成效 \\
\hline
第6週 & 第一次量表施測(引導期末)、反思報告(1) & 研究問題二:學習行為轉變 \\
\hline
第10週 & 反思報告(2) & 研究問題四:學習者觀點 \\
\hline
第12週 & 第二次量表施測(轉換期末)、焦點團體訪談、期中專題 & 研究問題二、三、四 \\
\hline
第14週 & 反思報告(3) & 研究問題四:學習者觀點 \\
\hline
第17週 & Python後測、第三次量表施測(自主期末) & 研究問題一、二 \\
\hline
第18週 & 期末專題、反思報告(4) & 研究問題一、四 \\
\hline
全學期 & AI互動紀錄 & 研究問題三:學習歷程分析 \\
\hline
\end{tabular}
\end{table}

\subsection{資料分析方法}

\subsubsection{量化資料分析}

\begin{enumerate}
\item \textbf{Python程式設計能力}(對應研究問題一):描述統計、成對樣本t檢定、效果量Cohen's d
\item \textbf{AI依賴與SRL能力轉變}(對應研究問題二):描述統計、重複量數變異數分析
\item \textbf{AI互動行為}(對應研究問題三):描述統計、各階段行為類型分布比較
\end{enumerate}

\subsubsection{質性資料分析}

針對學習反思與訪談資料(研究問題四),採用主題分析法進行編碼與詮釋\cite{braun2006thematic};雙人編碼30\%以提升一致性。

\subsubsection{資料整合}

採用方法三角檢證,整合量化與質性資料,從不同角度檢視研究問題。

\subsection{研究倫理}

\begin{enumerate}
\item \textbf{知情同意}:開學第一週向學生說明研究目的與資料使用方式,學生簽署知情同意書
\item \textbf{資料保護}:所有資料去識別化處理,僅供學術研究使用
\item \textbf{AI使用揭露}:參考相關建議\cite{becker2023programming},明確告知AI使用規範與學術誠信要求
\end{enumerate}
