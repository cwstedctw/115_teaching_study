%% ============================================
%% 115年度教育部教學實踐研究計畫
%% 伍、計畫主持人
%% ============================================

\section{計畫主持人}

\subsection{計畫主持人相關教學經歷}

在近三年的教學實務中,本人主要於國立東華大學通識教育中心規劃並開設多班「初級程式設計-Python」課程,以非資訊背景學生為主要對象,從程式思維與基礎語法出發,帶領學生熟悉變數、條件判斷、迴圈、函式與基本資料結構,並透過實務題目與小型專題,培養其運用程式解決問題的能力。

近三年Python課程開設紀錄如表~\ref{tab:python_courses}所示。

\begin{table}[htbp]
\centering
\caption{近三年「初級程式設計-Python」課程開設紀錄}
\label{tab:python_courses}
\begin{tabular}{|c|c|c|c|c|c|}
\hline
\textbf{學年度} & \textbf{學期} & \textbf{班別} & \textbf{教學評量} & \textbf{修課人數} & \textbf{學院平均} \\
\hline
112 & 1 & AC班 & 4.23 & 38 & 4.33 \\
\hline
112 & 2 & AA班 & 4.42 & 50 & 4.16 \\
\hline
113 & 1 & AC班 & 4.21 & 49 & 4.21 \\
\hline
\multicolumn{3}{|c|}{\textbf{平均}} & \textbf{4.29} & \textbf{總計137人} & --- \\
\hline
\end{tabular}
\end{table}

\subsection{教學特色與成效}

綜合近三年教學紀錄,本人共開設3班次Python課程,累計修課人數達137人,平均教學評量4.29分,顯示教學品質穩定且具成效。

本人教學特色與成效主要體現在以下三個方面:
\begin{enumerate}
\item \textbf{深入理解學習困難}:三年累計教授137位非資訊背景學生學習Python,對其常見迷思概念、學習障礙與動機問題有第一手觀察與因應經驗。

\item \textbf{AI融入教學經驗}:同時開設「人工智慧概論」課程,熟悉生成式AI工具的教學應用與潛在風險,為本計畫之AI鷹架設計提供實務基礎。

\item \textbf{持續改進導向}:教學評量穩定維持在4.2分以上,顯示具備教學反思與持續優化的習慣,符合教學實踐研究之精神。
\end{enumerate}

綜上所述,本人之教學經歷與專業背景,為本研究計畫之執行奠定堅實基礎。
