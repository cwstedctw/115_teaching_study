%% ============================================
%% 115年度教育部教學實踐研究計畫
%% 伍、計畫主持人
%% ============================================

\section{計畫主持人}

\subsection{計畫主持人相關教學經歷}

在111至113學年度期間,本人於國立東華大學通識教育中心擔任專任教師,主要開設程式設計與人工智慧相關通識課程。教學對象涵蓋全校各學院非資訊背景學生,累計開設22門課程,修課總人數達800人,平均教學評量4.27分。

本人教學範疇涵蓋以下四大課程類型:

\begin{enumerate}
\item \textbf{初級程式設計-Python}:針對零基礎學生,從程式思維與基礎語法出發,培養運算思維與問題解決能力。

\item \textbf{中級程式設計-虛擬實境}:引導學生運用程式設計知識進行3D互動內容創作,結合跨域應用。

\item \textbf{人工智慧概論}:介紹AI基本原理與應用,培養學生對AI技術的理解與批判思考能力。

\item \textbf{創客入門-智慧生活裝置實作}:整合程式設計、電子電路與實作能力,進行IoT專題製作。

\item \textbf{科技藝術概論}:跨域結合科技與藝術,探索數位創作的可能性(藝術學院)。
\end{enumerate}

近三年完整教學紀錄如表~\ref{tab:all_courses}所示。

\begin{table}[htbp]
\centering
\small
\caption{111-113學年度完整教學紀錄(按課程類型分類)}
\label{tab:all_courses}
\tiny
\setlength{\tabcolsep}{2.5pt}
\begin{tabular}{|l|c|c|c|c|c|c|}
\hline
\textbf{課程名稱} & \textbf{學年度} & \textbf{學期} & \textbf{班別} & \textbf{評量} & \textbf{人數} & \textbf{該學期平均} \\
\hline
\multicolumn{7}{|c|}{\textbf{初級程式設計-Python(5班次,共227人)}} \\
\hline
初級程式設計-Python & 111 & 1 & AA & 4.14 & 51 & 4.22 \\
初級程式設計-Python & 111 & 2 & AA & 4.03 & 39 & 4.30 \\
初級程式設計-Python & 112 & 1 & AC & 4.23 & 38 & 4.33 \\
初級程式設計-Python & 112 & 2 & AA & 4.42 & 50 & 4.16 \\
初級程式設計-Python & 113 & 1 & AC & 4.21 & 49 & 4.21 \\
\hline
\multicolumn{4}{|c|}{\textbf{小計平均}} & \textbf{4.21} & \textbf{227} & --- \\
\hline
\multicolumn{7}{|c|}{\textbf{中級程式設計-虛擬實境(7班次,共187人)}} \\
\hline
中級程式設計-虛擬實境 & 111 & 1 & AA & 4.24 & 29 & 4.22 \\
中級程式設計-虛擬實境 & 111 & 1 & AB & 4.05 & 30 & 4.22 \\
中級程式設計-虛擬實境 & 111 & 1 & AC & 4.10 & 23 & 4.22 \\
中級程式設計-虛擬實境 & 111 & 2 & AA & 4.50 & 22 & 4.30 \\
中級程式設計-虛擬實境 & 111 & 2 & AB & 4.11 & 29 & 4.30 \\
中級程式設計-虛擬實境 & 112 & 1 & AA & 4.36 & 25 & 4.33 \\
中級程式設計-虛擬實境 & 112 & 1 & AB & 4.61 & 29 & 4.33 \\
\hline
\multicolumn{4}{|c|}{\textbf{小計平均}} & \textbf{4.28} & \textbf{187} & --- \\
\hline
\multicolumn{7}{|c|}{\textbf{人工智慧概論(6班次,共256人)}} \\
\hline
人工智慧概論 & 111 & 1 & --- & 4.28 & 40 & 4.22 \\
人工智慧概論 & 111 & 2 & AA & 4.53 & 37 & 4.30 \\
人工智慧概論 & 112 & 1 & AB & 4.25 & 38 & 4.33 \\
人工智慧概論 & 112 & 2 & AA & 4.09 & 56 & 4.16 \\
人工智慧概論 & 112 & 2 & AB & 3.99 & 35 & 4.16 \\
人工智慧概論 & 113 & 1 & AA & 4.20 & 50 & 4.21 \\
\hline
\multicolumn{4}{|c|}{\textbf{小計平均}} & \textbf{4.22} & \textbf{256} & --- \\
\hline
\multicolumn{7}{|c|}{\textbf{創客入門-智慧生活裝置實作(3班次,共90人)}} \\
\hline
創客入門-智慧生活裝置實作 & 111 & 1 & --- & 4.52 & 27 & 4.22 \\
創客入門-智慧生活裝置實作 & 111 & 2 & --- & 4.36 & 33 & 4.30 \\
創客入門-智慧生活裝置實作 & 112 & 1 & --- & 4.19 & 30 & 4.33 \\
\hline
\multicolumn{4}{|c|}{\textbf{小計平均}} & \textbf{4.36} & \textbf{90} & --- \\
\hline
\multicolumn{7}{|c|}{\textbf{科技藝術概論(1班次,共40人)}} \\
\hline
科技藝術概論 & 113 & 2 & --- & 4.31 & 40 & 4.31 \\
\hline
\multicolumn{4}{|c|}{\textbf{小計平均}} & \textbf{4.31} & \textbf{40} & --- \\
\hline
\hline
\multicolumn{4}{|c|}{\textbf{總計/整體平均}} & \textbf{4.27} & \textbf{800} & \textbf{4.25} \\
\hline
\end{tabular}
\end{table}

\subsection{教學特色與成效}

綜合111-113學年度教學紀錄,本人共開設22門課程,累計修課人數達800人,平均教學評量4.27分。其中「創客入門」課程(4.36分)與「中級程式設計-虛擬實境」課程(4.28分)表現穩定,但「初級程式設計-Python」課程評量(4.21分)仍有改善空間,需透過更有效的教學設計與AI鷹架支持以提升學生學習成效。

本人教學特色與成效主要體現在以下四個方面:

\begin{enumerate}
\item \textbf{深入理解學習困難}:三年累計教授227位非資訊背景學生學習Python(5班次),對其常見迷思概念、學習障礙與動機問題有第一手觀察與因應經驗,為本計畫之教學設計提供紮實基礎。

\item \textbf{AI融入教學經驗}:開設6班次「人工智慧概論」課程(累計256人),熟悉生成式AI工具的教學應用、學習影響與潛在風險,為本計畫之AI鷹架設計提供實務基礎。

\item \textbf{跨域整合能力}:除程式設計外,同時開設虛擬實境、創客實作與科技藝術等跨域課程,具備整合不同領域知識進行創新教學設計的能力。

\item \textbf{大班教學經驗}:曾執行50-56人之大班教學(112-2人工智慧概論),對於異質性學習群體的班級經營、差異化教學與學習支持策略有實務經驗。

\item \textbf{持續改進導向}:教學評量穩定維持在4.0分以上,22門課程中有19門達4.2分以上,顯示具備教學反思與持續優化的習慣,符合教學實踐研究之精神。
\end{enumerate}

\subsection{與本計畫之關聯性}

本人之教學經歷與專業背景,與本研究計畫具有高度關聯性:

\begin{itemize}
\item \textbf{目標學生群體經驗}:累計5班次、227人次之Python教學經驗,深入理解非資訊背景學生在程式學習上的困境與需求。

\item \textbf{AI教學應用基礎}:6班次、256人次之AI概論教學經驗,熟悉AI工具在教學中的應用模式與學習影響。

\item \textbf{實踐研究能力}:持續三年穩定之教學表現(4.27分),展現教學反思、設計優化與成效評估能力。

\item \textbf{跨域教學視野}:涵蓋程式設計、AI、虛擬實境、創客實作等多元課程,具備整合不同教學策略與評量方法的經驗。
\end{itemize}

綜上所述,本人之教學經歷與專業背景,為本研究計畫之執行奠定堅實基礎。
