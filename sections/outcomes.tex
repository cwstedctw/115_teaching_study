%% ============================================
%% 115年度教育部教學實踐研究計畫
%% 伍、預期完成工作項目與成果
%% ============================================

\subsubsection{預期完成工作項目與成果}

\paragraph{研究執行時程}

\begin{longtable}{|c|m{6cm}|m{5.5cm}|}
\hline
\textbf{期程} & \textbf{工作項目} & \textbf{預期產出} \\
\hline
\endfirsthead
\hline
\textbf{期程} & \textbf{工作項目} & \textbf{預期產出} \\
\hline
\endhead
\multicolumn{3}{|c|}{\textbf{準備階段(115年8月-116年1月)}} \\
\hline
115年8-9月 & 
\begin{minipage}[t]{\linewidth}
\vspace{1mm}
$\bullet$ 開發AI學習平台\\
$\bullet$ 製作18週課程教材
\vspace{1mm}
\end{minipage}
& 
\begin{minipage}[t]{\linewidth}
\vspace{1mm}
AI平台beta版\\
教材初稿
\vspace{1mm}
\end{minipage} \\
\hline
115年10-12月 & 
\begin{minipage}[t]{\linewidth}
\vspace{1mm}
$\bullet$ 研究工具預試與修訂\\
$\bullet$ AI平台測試\\
$\bullet$ 教材專家審查
\vspace{1mm}
\end{minipage}
& 
\begin{minipage}[t]{\linewidth}
\vspace{1mm}
驗證後研究工具\\
AI平台正式版\\
教材定稿
\vspace{1mm}
\end{minipage} \\
\hline
116年1月 & 
\begin{minipage}[t]{\linewidth}
\vspace{1mm}
$\bullet$ 課程網站建置\\
$\bullet$ 施測準備
\vspace{1mm}
\end{minipage}
& 
\begin{minipage}[t]{\linewidth}
\vspace{1mm}
課程網站上線\\
施測流程確認
\vspace{1mm}
\end{minipage} \\
\hline
\multicolumn{3}{|c|}{\textbf{實施階段(116年2月-6月)}} \\
\hline
116年2-3月 & 
\begin{minipage}[t]{\linewidth}
\vspace{1mm}
$\bullet$ 執行引導期教學(1-6週)\\
$\bullet$ 前測、第一次量表施測、反思報告(1)
\vspace{1mm}
\end{minipage}
& 
\begin{minipage}[t]{\linewidth}
\vspace{1mm}
前測資料\\
第一次量表資料\\
反思報告(1)
\vspace{1mm}
\end{minipage} \\
\hline
116年4月 & 
\begin{minipage}[t]{\linewidth}
\vspace{1mm}
$\bullet$ 執行轉換期教學(7-12週)\\
$\bullet$ 第二次量表施測、訪談、期中專題
\vspace{1mm}
\end{minipage}
& 
\begin{minipage}[t]{\linewidth}
\vspace{1mm}
第二次量表資料\\
訪談逐字稿\\
期中專題成果
\vspace{1mm}
\end{minipage} \\
\hline
116年5-6月 & 
\begin{minipage}[t]{\linewidth}
\vspace{1mm}
$\bullet$ 執行自主期教學(13-18週)\\
$\bullet$ 後測、第三次量表施測、期末專題
\vspace{1mm}
\end{minipage}
& 
\begin{minipage}[t]{\linewidth}
\vspace{1mm}
後測資料\\
第三次量表資料\\
期末專題成果
\vspace{1mm}
\end{minipage} \\
\hline
\multicolumn{3}{|c|}{\textbf{分析撰寫階段(116年6-7月)}} \\
\hline
116年6-7月 & 
\begin{minipage}[t]{\linewidth}
\vspace{1mm}
$\bullet$ 資料分析\\
$\bullet$ 撰寫研究報告\\
$\bullet$ 製作教學資源
\vspace{1mm}
\end{minipage}
& 
\begin{minipage}[t]{\linewidth}
\vspace{1mm}
成果報告\\
論文初稿\\
教學資源包
\vspace{1mm}
\end{minipage} \\
\hline
\caption{研究執行時程}
\end{longtable}

\paragraph{預期具體成果與KPI}

\paragraph{教學成效KPI}

\begin{table}[h]
\centering
\caption{教學成效KPI指標}
\renewcommand{\arraystretch}{1.25}
\begin{tabular}{|m{4cm}|m{5cm}|m{4cm}|}
\hline
\textbf{KPI項目} & \textbf{衡量指標} & \textbf{預期目標} \\
\hline
程式設計能力提升 & 後測成績顯著高於前測 & t檢定達顯著($p<.05$)、Cohen's d $\geq$ 0.5 \\
\hline
AI依賴程度下降 & 第三次量表分數低於第一次 & 重複量數ANOVA達顯著 \\
\hline
SRL能力提升 & 第三次量表分數高於第一次 & 重複量數ANOVA達顯著 \\
\hline
期末專題完成率 & 獨立完成期末專題比例 & $\geq$ 70\%(優於113學年31\%) \\
\hline
課程及格率 & 期末成績及格比例 & $\geq$ 75\%(優於113學年47\%) \\
\hline
\end{tabular}
\end{table}

\paragraph{研究產出}

\begin{enumerate}
\item \textbf{教學模組}
  \begin{itemize}
  \item 18週完整課程教材(簡報、講義、影片)
  \item PBL任務與評量規準(含臺灣開放資料應用)
  \item 三階段AI提示範本庫
  \end{itemize}

\item \textbf{研究工具}
  \begin{itemize}
  \item Python程式設計能力測驗(含信效度資料)
  \item AI依賴與SRL量表中文版
  \end{itemize}

\item \textbf{研究發表}
  \begin{itemize}
  \item 投稿《教學實踐與研究》期刊1篇
  \item 校內教學成果發表1場
  \end{itemize}

\item \textbf{開放資源}
  \begin{itemize}
  \item GitHub開源專案(課程教材、AI提示範本)
  \end{itemize}
\end{enumerate}

\paragraph{預期影響與貢獻}

\paragraph{學術貢獻}

\begin{enumerate}
\item \textbf{理論貢獻}:發展「AI鷹架漸退教學模式」,整合既有理論脈絡,如認知學徒制\cite{collins1989cognitive}與SRL理論\cite{zimmerman2002becoming}
\item \textbf{實證貢獻}:提供臺灣高教脈絡下AI輔助程式設計教學的實證資料,以回應相關文獻所指出之本土研究不足\cite{lai2020analysis}
\end{enumerate}

\paragraph{實務貢獻}

\begin{enumerate}
\item \textbf{教學創新}:參考近期系統性研究之啟發\cite{kazemitabaar2024codeaid},發展可操作的AI鷹架設計原則
\item \textbf{學生學習}:以自我決定理論為基礎\cite{ryan2000selfdetermination},透過AI鷹架漸退提升學生自主性
\item \textbf{推廣應用}:開放教學資源供其他教師參考,並回應相關論述所提出的AI時代教育挑戰\cite{becker2023programming}
\end{enumerate}

\paragraph{社會影響}

\begin{enumerate}
\item 回應運算思維素養之培育需求\cite{wing2006computational}
\item 以負責任AI原則為依據\cite{dignum2019responsible},建立學生正確使用AI的態度
\end{enumerate}
