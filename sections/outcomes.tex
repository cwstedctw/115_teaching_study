%% ============================================
%% 115年度教育部教學實踐研究計畫
%% 伍、預期完成工作項目與成果
%% ============================================

\subsection{預期完成工作項目與成果}

\subsubsection{研究執行時程}

\begin{table}[htbp]
\centering
\scriptsize
\caption{研究執行時程}
\label{tab:timeline}
\renewcommand{\arraystretch}{1.2}
\begin{tabular}{|c|p{5cm}|p{4cm}|}
\hline
\textbf{期程} & \textbf{工作項目} & \textbf{預期產出} \\
\hline
\multicolumn{3}{|c|}{\textbf{準備階段(115年8月--116年1月)}} \\
\hline
115年8--9月 & 
開發AI學習平台、製作18週課程教材
& 
AI平台beta版、教材初稿 \\
\hline
115年10--12月 & 
研究工具預試與修訂、AI平台測試、教材專家審查
& 
驗證後研究工具、AI平台正式版 \\
\hline
116年1月 & 
課程網站建置、施測準備
& 
課程網站上線 \\
\hline
\multicolumn{3}{|c|}{\textbf{實施階段(116年2月--6月)}} \\
\hline
116年2--3月 & 
執行引導期教學(1--6週)、前測與第一次量表施測
& 
前測資料、第一次量表資料 \\
\hline
116年4月 & 
執行轉換期教學(7--12週)、第二次量表施測與訪談
& 
第二次量表資料、訪談逐字稿 \\
\hline
116年5--6月 & 
執行自主期教學(13--18週)、後測與第三次量表施測
& 
後測資料、期末專題成果 \\
\hline
\multicolumn{3}{|c|}{\textbf{分析撰寫階段(116年6--7月)}} \\
\hline
116年6--7月 & 
資料分析、撰寫研究報告
& 
成果報告、論文初稿 \\
\hline
\end{tabular}
\end{table}

\subsubsection{預期成果與KPI}

\textbf{教學成效指標}:

\begin{table}[htbp]
\centering
\scriptsize
\caption{教學成效KPI指標}
\label{tab:kpi}
\renewcommand{\arraystretch}{1.1}
\begin{tabular}{|m{3cm}|m{4cm}|m{3.5cm}|}
\hline
\textbf{KPI項目} & \textbf{衡量指標} & \textbf{預期目標} \\
\hline
程式設計能力提升 & 後測成績高於前測 & 達統計顯著差異 \\
\hline
AI依賴程度下降 & 第三次量表分數低於第一次 & 呈現下降趨勢 \\
\hline
SRL能力提升 & 第三次量表分數高於第一次 & 呈現上升趨勢 \\
\hline
期末專題完成率 & 獨立完成期末專題比例 & $\geq$ 70\% \\
\hline
課程及格率 & 期末成績及格比例 & $\geq$ 75\% \\
\hline
\end{tabular}
\end{table}

\textbf{研究產出}:
\begin{itemize}[leftmargin=2em]
\item \textbf{教學模組}:18週完整課程教材、PBL任務與評量規準、三階段AI提示範本庫
\item \textbf{研究工具}:Python程式設計能力測驗、AI依賴與SRL量表中文版
\item \textbf{研究發表}:投稿《教學實踐與研究》期刊1篇、校內教學成果發表1場
\item \textbf{開放資源}:GitHub開源專案(課程教材、AI提示範本)
\end{itemize}

\subsubsection{預期貢獻}

\textbf{學術貢獻}:
\begin{itemize}[leftmargin=2em]
\item 發展「AI鷹架漸退教學模式」,整合認知學徒制\cite{collins1989cognitive}與SRL理論\cite{zimmerman2002becoming}
\item 提供臺灣高教脈絡下AI輔助程式設計教學的實證資料\cite{lai2020analysis}
\end{itemize}

\textbf{實務貢獻}:
\begin{itemize}[leftmargin=2em]
\item 發展可操作的AI鷹架設計原則\cite{kazemitabaar2024codeaid}
\item 透過AI鷹架漸退提升學生自主性\cite{ryan2000selfdetermination}
\item 開放教學資源供其他教師參考\cite{becker2023programming}
\end{itemize}
