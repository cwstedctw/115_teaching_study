\subsection{文獻探討}

本章將文獻探討分為三個層次:第一層討論學習問題與理論基礎,對應研究背景中提出的三大挑戰;
第二層討論AI鷹架與教學方法,提供本研究教學設計的理論基礎;
第三層分析現有研究的不足與研究缺口,說明本研究的必要性與獨特貢獻。

\paragraph{第一層:學習問題與理論基礎}

\paragraph{非資訊科系學生的程式設計學習困難}

隨著運算思維被視為21世紀核心素養,程式設計教育已從資訊科系專業擴展至通識教育領域。
然而,非資訊背景學生在學習過程中面臨諸多挑戰,這些困難形成了本研究「挑戰一」的背景。
Guzdial在其《以學習者為中心的運算教育設計》一書中\cite{guzdial2015learner}指出,
程式設計對非專業學習者而言是一項高度複雜的認知任務,需要特別設計的教學策略以降低學習障礙。
Qian與Lehman對程式設計學習困難進行大規模文獻分析,其\cite{qian2017misconceptions}系統性回顧文獻發現,
新手程式設計者面臨三大類困難:(1)語法知識(syntactic knowledge)錯誤,如括號不配對、變數命名規則誤用;
(2)概念知識(conceptual knowledge)誤解,如對迴圈、遞迴等核心概念的錯誤理解;
(3)策略知識(strategic knowledge)缺乏,即不知如何將問題分解為可編程的子任務。

從認知心理學角度來看,認知負荷理論(Cognitive Load Theory)提供了理解這些困難的理論框架\cite{guzdial2015learner}。
程式設計學習需要同時處理多重認知任務:理解抽象概念(如物件、函式、資料結構)、
記憶語法規則(如Python的縮排、關鍵字、運算子)、追蹤程式執行流程(如迴圈迭代、條件分支)、
偵錯與問題解決(如閱讀錯誤訊息、使用除錯工具)。這些任務對工作記憶造成巨大負擔,
特別是對缺乏先備知識的初學者。

\paragraph{生成式AI帶來的機會與挑戰}

2022年以來,生成式AI技術的快速發展對程式設計教育帶來巨大衝擊。
Finnie-Ansley及其研究團隊針對OpenAI Codex進行的實證研究\cite{finnieansley2022robots}顯示,
生成式AI工具(如GitHub Copilot、ChatGPT)能夠以約78\%的準確率解決CS1等級的程式設計問題,
甚至能處理複雜的編程作業。這看似為程式設計教育帶來了解決方案,卻同時引發新的教育挑戰
(對應本研究「挑戰二」:認知卸載危機):

\begin{itemize}
\item \textbf{過度依賴風險}:Kazemitabaar等學者研究發現\cite{kazemitabaar2023studying},
學生在使用AI程式碼生成器時,常直接複製AI生成的程式碼而不加修改、不進行理解驗證,
這種「複製-貼上」(copy-paste)的行為模式可能導致學習流於表面。
\item \textbf{錯誤接受率}:Prather及其同事的質性研究\cite{prather2024weird}觀察到,
當AI提供錯誤或不完整的答案時,初學者往往直接接受而不進行批判性思考,
顯示出學生對AI的「過度信任」(overtrust)問題。
\item \textbf{學習阻礙}:Finnie-Ansley等人\cite{finnieansley2022robots}警告,
若學生過度依賴AI生成完整解答而跳過自主思考過程,可能阻礙其建立紮實的程式設計基礎,
長期而言不利於培養獨立問題解決能力。Finnie-Ansley團隊後續研究\cite{finnieansley2023exam}
進一步發現,AI工具在CS2進階課程考試中同樣表現優異,能超越多數學生,這使得依賴問題更加嚴重。
\end{itemize}

\paragraph{傳統教學法的效能與侷限}

面對前述兩類挑戰(學習困難與AI依賴),傳統的講授式教學法顯得力不從心
(對應本研究「挑戰三」:傳統教學方法的侷限)。
Hmelo-Silver\cite{hmelo2004problem}指出,單向講授模式難以促進學生的深度學習與自主學習能力,
特別是在需要應用知識解決複雜問題的程式設計學科。
臺灣本土研究也證實這個現象:林育慈與陳志洪\cite{lin2021teaching}以及賴阿福與林志隆\cite{lai2020analysis}
的研究都發現,傳統講授模式下學生的學習焦慮高、學習動機低,
需要更積極、更以學習者為中心的教學方法。

\paragraph{第二層:AI鷹架與教學方法}

\paragraph{AI鷹架設計原則}

為了平衡AI工具的協助性與學習效果,近期研究開始探索「結構化AI鷹架」的設計原則。
Kazemitabaar研究團隊\cite{kazemitabaar2024codeaid}開發的CodeAid系統,
採用「不直接提供完整程式碼」的設計哲學,反而透過概念解釋(解釋程式設計原理)、
虛擬碼生成(提供演算法框架)、錯誤標註(標示問題所在位置)等方式提供分層級的學習支持。
經過一學期的課堂實驗,研究顯示此種結構化AI鷹架能有效促進學生的深度學習,
同時避免過度依賴,且獲得教師與學生的高度評價。

從另一個角度,Denny等學者在\cite{denny2023conversing}的研究指出,
提示工程(prompt engineering)能力本身就是一項重要的學習目標與未來技能。
學生學習如何精確描述問題(明確需求、提供上下文)、修正提示詞(根據AI回應調整問法)以獲得更好回應,
此迭代互動的過程本身就培養了運算思維、問題分解與溝通表達能力。

然而,現有AI鷹架研究多採用「靜態」設計(固定程度的支持),
缺乏「動態」調整機制(根據學習進展調整支持程度)。
本研究提出的「三階段AI鷹架漸退模式」在此具有研究創新性,
將鷹架支持從高(引導期)逐步降低(轉換期)直至最小化(自主期),
配合學生能力發展軌跡。

\paragraph{自我調節學習(SRL)理論}

自我調節學習(Self-Regulated Learning, SRL)理論為本研究的教學設計提供了重要的理論基礎。
Zimmerman在其影響深遠的研究\cite{zimmerman2002becoming}中將自我調節學習定義為
學習者主動設定學習目標(目標設定)、監控學習歷程(後設認知監控)、
調整學習策略(策略選擇與調適)的循環過程。這三個階段形成一個動態的回饋迴圈,
促進學習者持續改進其學習表現。
Azevedo與Gašević針對科技輔助學習環境的研究\cite{azevedo2019using}進一步指出,
在使用進階學習技術(如AI工具)的環境中,SRL能力是學習成功的關鍵預測因子,
缺乏SRL能力的學生容易陷入被動學習與過度依賴的困境。

Järvelä與Hadwin在協作學習的脈絡下\cite{jarvela2015enhancing}提出社會性共享調節(socially shared regulation)的概念,
強調在協作學習中,學習者如何透過互動溝通、共同計畫、互相監督等方式共同調節學習歷程,
這對本研究中小組協作學習環節的設計具有啟發意義。

SRL理論直接對應本研究的「研究問題二」(學習行為轉變),
探討學生如何從低度SRL能力(依賴AI)逐步發展至高度SRL能力(自主學習),
以及AI鷹架漸退如何促進這個轉變過程。

\paragraph{問題導向學習(PBL)}

問題導向學習(Problem-Based Learning, PBL)是另一個重要的教學法框架。
Savery在對跨學科PBL文獻的綜合回顧\cite{savery2006overview}中定義PBL為
「以真實世界的複雜問題為起點,學習者透過自主探究、資料搜集與協作討論來解決問題的教學法」。
此教學法強調學習者主動性與知識建構,有別於傳統的講授式教學。
Hmelo-Silver的實證研究\cite{hmelo2004problem}顯示,PBL能有效促進深度學習(超越表面記憶的理解)、
問題解決能力(分析、綜合、評估技能)與自主學習動機(內在動機與持續性),
特別適合於程式設計這類需要應用知識解決實際問題的學科。

在臺灣本土的脈絡下,教學實踐研究也提供了PBL在程式設計教學成效的實證證據。
林育慈與陳志洪的教學實踐研究\cite{lin2021teaching}將PBL與翻轉教室結合應用於Python程式設計課程,
經過一學期的實驗教學,結果顯示實驗組學生在學習成效、學習動機與學習滿意度三個面向都顯著優於對照組。
賴阿福與林志隆針對非資訊科系大學生的研究\cite{lai2020analysis}發現,
問題導向教學模式能顯著降低學生的程式設計學習焦慮,提升其學習成就與自我效能,
這對本研究針對通識課程非本科系學生的教學設計具有直接參考價值。

本研究將PBL的精神融入AI鷹架漸退設計:每階段都以真實的程式設計問題為導向,
鼓勵學生主動探索解決方案,而AI鷹架的角色則從「引導者」逐步轉變為「協助者」再到「後援資源」。

\paragraph{動機理論與學習投入}

學習動機是影響學習成效的關鍵因素。Ryan與Deci的自我決定理論(Self-Determination Theory)\cite{ryan2000selfdetermination}
指出,內在動機(源自興趣、好奇心與挑戰)相較於外在動機(獲得獎勵、避免懲罰)
更能促進深度學習與持續投入。該理論提出三個基本心理需求:
自主性(autonomy)、能力感(competence)與關係感(relatedness)。

本研究的AI鷹架漸退設計旨在滿足這三個需求:透過逐步減少外部支持提升自主性,
透過適當的鷹架支持保證能力感(避免過高挫敗),透過小組協作建立關係感。
這對應本研究的「預期效益」中提到的「促進內在學習動機」。

\paragraph{認知學徒制(Cognitive Apprenticeship)}

認知學徒制(Cognitive Apprenticeship)理論則為本研究的「AI鷹架漸退」機制提供了直接的理論基礎。
Collins, Brown與Newman在其開創性的論文\cite{collins1989cognitive}中提出的認知學徒制模式,
強調透過「示範(modeling)-指導(coaching)-鷹架(scaffolding)-逐步退出(fading)」四個階段的有系統過程,
將專家的隱性認知歷程(如思考策略、問題解決步驟、決策機制)外顯化給學習者。
其中「鷹架漸退」(scaffolding fading)的核心精神是:初期提供大量支持,
隨著學習者能力增長逐步減少外部協助,最終促使學習者內化知識與技能、達到獨立運作。
此理論框架直接啟發了本研究的AI鷹架三階段漸退設計。

\paragraph{第三層:教學設計與研究缺口}

\paragraph{如何設計有效的AI鷹架以促進學習}

綜合上述文獻,可以看出現有研究主要聚焦於生成式AI工具的能力評估與影響分析。
例如Finnie-Ansley及其同事\cite{finnieansley2022robots}與Becker研究團隊\cite{becker2023programming}的研究多探討
「AI能做什麼」、「學生如何使用AI」、「使用AI對成績的影響」等議題,
但較少深入探討如何在教學中有效、系統化地整合AI工具以促進真正的深度學習。
Kazemitabaar團隊的CodeAid研究\cite{kazemitabaar2024codeaid}是重要的突破,其研究顯示精心設計的AI鷹架
(如分層提示、漸進式提示)能在提供協助與促進學習之間取得平衡。

然而,即使是CodeAid研究也主要採用固定結構的鷹架設計,
缺乏長期(全學期或跨學期)、系統性的縱貫性教學設計研究,
探討如何透過階段性、有計畫地調整AI鷹架的支持程度(從高到低),
並配合教學策略與學習活動,培養學生從AI依賴逐步轉變為AI協作、最終達到自主學習的能力。

\paragraph{研究缺口與本研究的獨特貢獻}

綜合以上文獻回顧,可識別出以下五個主要研究缺口,以及本研究如何填補這些缺口:

\begin{enumerate}
\item \textbf{缺乏階段性、縱貫性的AI鷹架研究}:
現有研究(如CodeAid\cite{kazemitabaar2024codeaid})多採用固定結構的AI支持,
缺乏長期(全學期)、系統性地探討如何根據學習進展調整AI鷹架程度。
\textbf{本研究貢獻}:首次提出系統化的「三階段AI鷹架漸退模式」,跨越18週教學實踐,
從高度支持(引導期)逐步退至最小支持(自主期)。

\item \textbf{SRL能力培養機制與轉變過程不明}:
少有研究深入探討AI輔助學習中,學生如何從依賴AI轉變為自主學習,
SRL能力的具體發展軌跡與影響因素為何。
\textbf{本研究貢獻}:結合SRL理論架構,追蹤學生在三階段中的SRL能力轉變
(對應研究問題二),探討具體表現與影響機制。

\item \textbf{學習歷程資料分析不足}:
現有研究多依賴成績與問卷等最終產出(outcome)數據,
缺乏對學習過程(process)的細緻分析。
\textbf{本研究貢獻}:結合學習歷程資料(學生與AI互動記錄、編程歷程),
深入分析學生的學習策略、認知負荷與問題解決歷程(對應研究問題三)。

\item \textbf{臺灣本土實證研究不足}:
臺灣針對非資訊科系學生的AI輔助程式設計教學研究仍相當有限,
特別是在通識教育脈絡下的實證研究。
\textbf{本研究貢獻}:以國立東華大學通識課程為研究情境,
提供本土化的教學設計與實施經驗,並探討文化與教育脈絡對教學成效的影響。

\item \textbf{教學實踐證據與可操作性不足}:
大部分研究停留在理論探討或小規模實驗,
缺乏來自真實課堂的長期教學實踐證據與可移轉的教學模式。
\textbf{本研究貢獻}:透過一學期(18週)的實際教學實踐,
提供詳細的教學設計、實施步驟、遇到的挑戰與解決方案,
以便其他教育工作者參考與移轉(對應預期效益中的「可推廣至其他課程」)。
\end{enumerate}

綜上所述,本研究的獨特性在於:系統化的AI鷹架漸退機制、
SRL能力轉變追蹤、學習歷程深度分析、本土實證研究與長期教學實踐的結合,
五個面向的整合在國際上AI輔助程式設計教育研究中尚屬空白。
