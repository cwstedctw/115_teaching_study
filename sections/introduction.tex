
\section{研究動機與背景}

\subsection{教學問題脈絡}

隨著人工智慧技術的普及,程式設計能力已成為21世紀公民的基本素養\cite{wing2006computational}。
本校通識教育中心開設之「程式設計與應用」課程,旨在培養非資訊科系學生的運算思維與程式設計能力。
然而,教學實踐中面臨三大核心挑戰:

\textbf{挑戰一:學習困難與高挫折率}

113學年度課程資料顯示,62\%學生反映初期學習困難,期中考僅47\%及格,且僅31\%學生能獨立完成期末專題。
非資訊背景學生在學習程式設計時面臨多重障礙,包括抽象概念理解困難、語法錯誤頻繁、
缺乏問題分解能力等\cite{guzdial2015learner,qian2017misconceptions}。

\textbf{挑戰二:生成式AI帶來的認知卸載危機}

ChatGPT等生成式AI工具能夠快速生成完整程式碼\cite{finnieansley2022robots,becker2023programming},
但學生過度依賴AI可能導致「認知卸載」(cognitive offloading),即將思考過程外包給AI,
而未真正理解程式邏輯\cite{kazemitabaar2023studying}。
研究顯示,當學生直接複製AI生成的程式碼而不加修改時,92\%會接受錯誤答案\cite{prather2024weird}。

\textbf{挑戰三:傳統教學方法的侷限}

傳統的「講述-練習」教學模式難以有效支持學生建立自主學習能力。
教師難以即時診斷每位學生的學習困難,也無法提供個別化的學習支持\cite{hmelo2004problem}。

\subsection{傳統教學方法說明}

過去本課程採用以下教學方法:

\begin{enumerate}
\item \textbf{講述法為主}:教師透過投影片講解程式概念與語法,學生被動接收知識。
\item \textbf{範例示範}:教師示範程式碼撰寫過程,學生模仿練習。
\item \textbf{課後作業}:指派固定題目供學生練習,但缺乏即時回饋機制。
\item \textbf{AI工具使用方式}:允許學生自由使用ChatGPT等工具,但未提供使用指引或限制。
\end{enumerate}

此模式的主要問題:
\begin{itemize}
\item 學生缺乏主動探索與問題解決的機會
\item 無法培養自我調節學習能力
\item AI工具使用缺乏教學設計,導致過度依賴
\item 評量方式難以區分學生真實能力與AI協助程度
\end{itemize}

\subsection{改進方向與預期效益}

本研究提出「AI鷹架漸退式教學模式」,
結合自我調節學習理論(Self-Regulated Learning, SRL)\cite{zimmerman2002becoming,azevedo2019using}
與問題導向學習(Problem-Based Learning, PBL)\cite{savery2006overview},
將AI從「答案提供者」轉變為「學習鷹架」。

\textbf{核心理念}:透過三階段AI鷹架漸退設計:
\begin{enumerate}
\item \textbf{引導期}(1-6週):高度結構化AI提示,引導學生思考
\item \textbf{轉換期}(7-12週):部分鷹架移除,鼓勵學生先嘗試
\item \textbf{自主期}(13-18週):最小化AI介入,學生主導學習
\end{enumerate}

\textbf{預期效益}:
\begin{itemize}
\item 提升學生程式設計學習成效與自我效能
\item 降低AI過度依賴,培養自我調節學習能力
\item 發展有效的AI輔助教學模式,可推廣至其他課程
\end{itemize}

\subsection{研究問題}

本研究擬探討以下四個核心問題:

\begin{enumerate}
\item 在三階段AI鷹架漸退模式下,學生的Python程式設計能力(知識理解、程式撰寫、問題解決)是否顯著提升?
\item 學生對AI工具的依賴程度與自我調節學習能力,在三階段教學中如何轉變?
\item 不同階段的AI鷹架設計,如何影響學生的學習歷程與行為模式?
\item 學生對AI輔助程式設計學習的態度、經驗與反思為何?
\end{enumerate}

